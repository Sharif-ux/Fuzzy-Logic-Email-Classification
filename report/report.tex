%%%%%%%%%%%%%%%%%%%%%%%%%%%%%%%%%%%%%%%%%%%%%%%%%%%%%%%%%%%%%%%%%%%%%
%
% Draft report Fuzzy Logic
%
%%%%%%%%%%%%%%%%%%%%%%%%%%%%%%%%%%%%%%%%%%%%%%%%%%%%%%%%%%%%%%%%%%%%%%

\documentclass[a4paper]{article}
\usepackage[utf8]{inputenc}
\usepackage[english]{babel}

\usepackage[myheadings]{fullpage}
\usepackage{fancyhdr}
\usepackage{lastpage}
\usepackage{graphicx, wrapfig, subcaption, setspace, booktabs}
\usepackage[font=small, labelfont=bf]{caption}
\usepackage{fourier}
\usepackage[protrusion=true, expansion=true]{microtype}

\usepackage{sectsty}

\newcommand{\HRule}[1]{\rule{\linewidth}{#1}}
\onehalfspacing
\setcounter{tocdepth}{5}
\setcounter{secnumdepth}{5}

\usepackage{amsmath}
\usepackage{hyperref}
\usepackage{lipsum}

%-------------------------------------------------------------------------------
% HEADER & FOOTER
%-------------------------------------------------------------------------------
\pagestyle{fancy}
\fancyhf{}
\setlength\headheight{15pt}
\fancyhead[L]{Peter Heemskerk, Stefan Schenk, Jim Kamans}
\fancyhead[R]{Draft Report}
\fancyfoot[R]{Page \thepage\ van \pageref{LastPage}}
%-------------------------------------------------------------------------------
% TITLE PAGE
%-------------------------------------------------------------------------------

\begin{document}

\title{ \normalsize \textsc{University of Amsterdam}
    \\ [2.0cm]
    \HRule{0.5pt} \\
    \LARGE \textbf{\uppercase{Draft Report Fuzzy Logic}}
    \HRule{2pt} \\ [0.5cm]
    \normalsize \today \vspace*{5\baselineskip}}

\date{}

\author{
    Authors: Peter Heemskerk, Stefan Schenk, Jim Kamans \\
    Studentnrs: \dots, 11881798, 10302905 \\
        Course: Fundamentals of Fuzzy Logic}

\maketitle
\newpage

\tableofcontents
\newpage

%-------------------------------------------------------------------------------
% Section title formatting
\sectionfont{\scshape}
%-------------------------------------------------------------------------------

%-------------------------------------------------------------------------------
% BODY
%-------------------------------------------------------------------------------
\section{Introduction}
\lipsum

\section{Objectives}
\lipsum

\section{Literature reviews}

\textit{Provide a coherent and synthesized summary of relevant work in the
literature, present the supporting/related evidence for your project.} \\

A Content Based Classification of Spam Mails with Fuzzy Word Ranking
\cite{spam} introduces

A Proactive Anti-Phishing Tool Using Fuzzy Logic and RIPPER Data Mining Classification Algorithm \cite{phishing}\\

A new fuzzy logic based ranking function for efficient Information Retrieval system \cite{ranking} explains how "conventional ranking functions
fail to capture inherent features of documents and queries due to subjectivity
involved in natural language text", and offers a fuzzy approach to rank words
based on term-weighting schema's such as term frequency, inverse document
frequency and normalization.

% \cite[chapter, p.~215]

\section{Approach}
In this section \dots

\subsection{Data}
\lipsum[3]


\subsection{Design}
\lipsum[2]


\subsection{Implementation}
\lipsum[1]

%-------------------------------------------------------------------------------
% BIBLIOGRAPHY
%
% EX:
% \cite{citation1}
%
% \bibitem{citation1}
%     Leslie Lamport,
%     \textit{\LaTeX: a document preparation system},
%     Addison Wesley, Massachusetts,
%     2nd edition,
%     1994.
%-------------------------------------------------------------------------------
\begin{thebibliography}{9}

\bibitem{spam}
    G. Santhi, S. Maria Wenish, Dr. P. Sengutuvan,
    \textit{
        \href{https://github.com/Menziess/Fuzzy-Logic-Email-Classification/raw/master/report/res/a_content_based_classification_of_spam_mails_with_fuzzy_word_ranking.pdf}{A Content Based Classification of Spam Mails with Fuzzy Word Ranking},
    }
    Department of Information Science and Technology,
    Issue 3,
    2013.

\bibitem{phishing}
    Rosana J. Ferolin,
    \textit{
        \href{https://github.com/Menziess/Fuzzy-Logic-Email-Classification/raw/master/report/res/a_proactive_anti-phishing_tool_using_fuzzy_logic_and_ripper_data_mining_classification_algorithm.pdf}{A Proactive Anti-Phishing Tool Using Fuzzy Logic and RIPPER Data Mining Classification Algorithm},
    }
    Department of Computer Engineering University of San Carlos.

\bibitem{ranking}
    Rosana J. Ferolin,
    \textit{
        \href{https://github.com/Menziess/Fuzzy-Logic-Email-Classification/raw/master/report/res/a_new_fuzzy_logic_based_ranking_function_for_efficient_information_retrieval_system.pdf}{A new fuzzy logic based ranking function for efficient Information Retrieval system},
    }
    Department of Electrical Engineering Dayalbagh Educational Institute,
    2014.

\end{thebibliography}

\end{document}
